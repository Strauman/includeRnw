%% includeRnw v0.0.1b3 - 2018/04/21
%% The LaTeX package includeRnw - version v0.0.1 (2018/04/21) build 3
%% Manual/Documentation for includeRnw.sty
%% -------------------------------------------------------------------------------------------
%% Copyright (c) 2018 by Andreas Storvik Strauman
%% -------------------------------------------------------------------------------------------
%% This work may be distributed and/or modified under the
%% conditions of the LaTeX Project Public License, either version 1.3c
%% of this license or (at your option) any later version.
%% The latest version of this license is in
%%   http://www.latex-project.org/lppl.txt
%% and version 1.3c or later is part of all distributions of LaTeX
%% version 2008/05/04 or later.
%% This work has the LPPL maintenance status `author-maintained'.
%% This work consists of all files listed in README.txt
\documentclass{article}
\usepackage[all]{tcolorbox}
\usepackage{needspace}
\usepackage{tabularx}
\makeatletter
\def\input@path{{../../docs/}}
\lstdefinestyle{mydocumentation}{style=tcbdocumentation,
  classoffset=0,
  texcsstyle=*\color{blue},
  moretexcs={arrayrulecolor,draw,includegraphics,ifthenelse,isodd,lipsum,path,pgfkeysalso},
  classoffset=1,
  moretexcs={% core
    includeRnw
  },
  texcsstyle=*\color{Definition}\bfseries,
  classoffset=0,% restore default
  }
\newtcolorbox{marker}[1][]{enhanced,
    before skip=2mm,after skip=3mm,
    boxrule=0.4pt,left=5mm,right=2mm,top=1mm,bottom=1mm,
    colback=yellow!50,
    colframe=yellow!20!black,
    sharp corners,rounded corners=southeast,arc is angular,arc=3mm,
    underlay={%
      \path[fill=tcbcol@back!80!black] ([yshift=3mm]interior.south east)--++(-0.4,-0.1)--++(0.1,-0.2);
      \path[draw=tcbcol@frame,shorten <=-0.05mm,shorten >=-0.05mm] ([yshift=3mm]interior.south east)--++(-0.4,-0.1)--++(0.1,-0.2);
      \path[fill=yellow!50!black,draw=none] (interior.south west) rectangle node[white]{\Huge\bfseries !} ([xshift=4mm]interior.north west);
      },
    drop fuzzy shadow,#1}
  \def\l@macro#1#2{#1\hfill\newline}
\newcommand\macrotable{\hypersetup{linkcolor=black}\@starttoc{mac}\hypersetup{linkcolor=Definition}}
\let\oldTOC\tableofcontents
\renewcommand\tableofcontents{\hypersetup{linkcolor=black}\oldTOC\hypersetup{linkcolor=Definition}}
\reversemarginpar
\def\updated#1{\tcbdocmarginnote{\bfseries{\color{blue}U}#1}}
\def\defnew#1{\tcbdocmarginnote{\bfseries{\color{green}N}#1}}
\makeatother
\let\dac\docAuxCommand
\long\def\keyDef#1#2#3#4{\begin{docKey}{#1}{=\meta{#2}}{\meta{default}=#3}#4\end{docKey}}
\long\def\optDef#1#2{\begin{docKey*}{#1}{}{}#2\end{docKey*}}
\tcbset{documentation listing style=mydocumentation}
% Magenta HREF style
\let\oldhref\href
\gdef\href#1#2{{\color{magenta}\oldhref{#1}{#2}}}
\tcbset{documentation listing style=mydocumentation,/tcb/color hyperlink=Definition}
\hypersetup{colorlinks=true}
% Give section some space
\let\oldsection\section
\gdef\section{\needspace{0.3\paperheight}\oldsection}
\let\oldsubsection\subsection
\gdef\subsection{\needspace{0.2\paperheight}\oldsubsection}


\setlength{\parindent}{0pt}
\title{{includeRnw - manual\\ v0.0.1{\\[-0.5em]\footnotesize(build 3)}}}
\author{Andreas Strauman}
\begin{document}
\maketitle
 
If you found any bugs or want new functionality, to contribute, view the commented source, get latest version of this package or get in touch with me, you can do all of that at \url{https://github.com/Strauman/includeRnw/}. If you have questions of functionality, kindly direct them to the community\\ \url{http://tex.stackexchange.com}. The author is active on this site regularly.

\tableofcontents
\clearpage
 \section{Reference}
\filbreak\subsection{Quick start}
\begin{docCommand}{includeRnw}{\oarg{h}\marg{path/to/file.Rnw}}\addcontentsline{toc}{subsubsection}{\refCom{includeRnw}}
\defnew{v0.0.1\\2018/04/20 }
Compiles .Rnw-files using R. Assumes that \texttt{R} can be called
from the command line. Use the \oarg{h} to prevent \dac{includeRnw} from
actually knitting the file, but only include the knitted \texttt{.tex} file
\end{docCommand}
\filbreak\subsection{Package Options}
\dac{usepackage}[\oarg{options}] where the \meta{options} are:\\
\optDef{halt}{If you do \dac{usepackage}[halt]\brackets{includeRnw},
then \dac{includeRnw}\{my/file.Rnw\} will \emph{not} run knitr on it.
However, if the knitted file exists, it will include this. You can
override this with the f-option in the \refCom{includeRnw}:
\dac{includeRnw}[f]{my/file.Rnw}}
\optDef{new}{This option would make the \refCom{includeRnw} only
run knitr on the file if the knitted file does not exist.
}

\section{Changelog}
v0.0.1 2018/04/20 Created the package
 \end{document}
