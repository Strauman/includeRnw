%!TEX root = main.tex
\RequirePackage{pdftexcmds}
\RequirePackage{kvoptions}
\makeatletter
\let\incl\input
\def\insp#1{\texttt{\string#1:\meaning#1}}
\def\inspw#1{\@latex@warning{\string#1:\meaning#1}}
% Log levels:
% 0: No logging (Should be used in production)
% 1: overall
% 2: verbose / info
% 3: very verbose
\providecommand\rnw@loglevel{3}
% dloglevels: log-levels+1
% 0: overall
% 1: verbose / info
% 2: very verbose
\newcommand\@dlog[2][1]{\ifnum\rnw@loglevel>#1\relax\@latex@warning{#2}\fi}
%% Package options
%:!iTOCOff
%:§popts
%:\usepackage[\oarg{options}] where the \meta{options} are:\\
%:\optDef{halt}{If you do \usepackage[halt]\brackets{includeRnw},
%:then \includeRnw{my/file.Rnw} will \emph{not} run knitr on it.
%:However, if the knitted file exists, it will include this. You can
%:override this with the f-option in the \refCom{includeRnw}:
%:\includeRnw[f]\brackets{my/file.Rnw}}
%:\optDef{new}{This option would make the \refCom{includeRnw} only
%:run knitr on the file if the knitted file does not exist.
%:}
%:\optDef{noknithead}{
%:This option will prevent \includeRnw from building and including the knitr-preamble.
%:}
%:\optDef{texenv}{
%:Using the \texttt{texenv} option disables the default Rnw syntax and allows for a new one. See example below
%:}
%: The two following examples do the exact same thing:\\
%: This example use standard Rnw:
%:!\documentclass[a4paper]{article}
%:!\usepackage{includeRnw}
%:!\begin{document}
%:! <<pChunk, echo=F, cache=F >>=
%:!   read_chunk('r/myRScript.r')
%:! @
%:!\end{document}
%:The following chunk uses the \refKey{texenv} option:
%:!\documentclass[a4paper]{article}
%:!\usepackage[texenv]{includeRnw}
%:!\begin{document}
%:! \begin{Rnw}{myChunk, echo=F, cache=F}
%:!   read_chunk('r/myRScript.r')
%:! \end{Rnw}
%:!\end{document}
%:-

\RequirePackage{kvoptions}
\SetupKeyvalOptions{
 family=includeRnw,
 prefix=rnw@opt@
}
\DeclareBoolOption[true]{build}
\DeclareBoolOption[true]{classicenv}
\DeclareComplementaryOption{halt}{build}
\DeclareBoolOption[false]{new}
\DeclareBoolOption[true]{knithead}
\DeclareComplementaryOption{noknithead}{knithead}
\DeclareComplementaryOption{texenv}{classicenv}
\ProcessKeyvalOptions*\relax
%------ Build commands ------%
\def\rnw@knitCommand@classic{%
R -e 'library("knitr");
knit("\rnw@dir@input/\rnw@infile@fullpath",  "\rnw@dir@output/\filename@base\rnw@suffix.tex")' &> \rnw@file@knitlog
}%
\def\rnw@knitCommand@texenv{%
% echo "knit_patterns\string\$set(list(
% chunk.begin = '\doublebs s*\fourbs begin\doublebs{Rnw\doublebs}\doublebs{([^{}]+)\doublebs}',
% chunk.end = '\doublebs s*\fourbs end\doublebs {Rnw\doublebs }'
% ))" > \rnw@dir@output/test.knit;
R -e "library('knitr');
knit_patterns\string\$set(list(
chunk.begin = '\doublebs s*\fourbs begin\doublebs{Rnw\doublebs}\doublebs{(.+)\doublebs}',
chunk.end = '\doublebs s*\fourbs end\doublebs {Rnw\doublebs}',
inline.code = '\fourbs Sexpr\doublebs{([^{}]+)\doublebs}',
ref.chunk = '\fourbs rCode\doublebs{([^{}]+)\doublebs}'
));
knit('\rnw@dir@input/\rnw@infile@fullpath',  '\rnw@dir@output/\filename@base\rnw@suffix.tex')" &> \rnw@file@knitlog%
}%
%------ Choosing build commands ------%
\ifrnw@opt@classicenv\relax
\global\let\rnw@knitCommand\rnw@knitCommand@classic
\@dlog[1]{Using classic environment}
\else
\@dlog[1]{Using tex environment}
\global\let\rnw@knitCommand\rnw@knitCommand@texenv
\fi

% Check if option clash
% if halt && new
\ifrnw@opt@build\else
  \ifrnw@opt@new
    \@latex@warning{@@PACKAGE: can't use halt and new together. Falling back to new.}
    \rnw@opt@haltfalse
  \fi
\fi

\providecommand\rnw@dir@input{.}
\providecommand\rnw@dir@output{./knitrout}
\providecommand\rnw@suffix{knitted}
\providecommand\rnw@file@knitlog{\rnw@dir@output/knitlog.log}
\providecommand\rnw@filebase@knithead{\rnw@dir@output/knithead}



\def\rnw@filename@parse#1{%
 \filename@parse{#1}
 \edef\rnw@filepath{\filename@area\filename@base}
 \edef\rnw@filebase{\filename@base}
 \edef\rnw@fileext{\ifx\filename@ext\relax Rnw\else\filename@ext\fi}
 \edef\rnw@infile@fullpath{\filename@area\filename@base.\filename@ext}
}
\def\rnw@clear@knitlog{%
\immediate\write18{echo "" > \rnw@file@knitlog}
}

\def\rnw@check@ouput@file{\rnw@dir@output/.includeRnwShellEscapeCheck}

\def\check@shell@escape{%
  \ifcase\pdf@shellescape%
  \PackageError{@@PACKAGE}{\string\includeRnw\space requires --shell-escape}{}\stop\or%
  \message{Shell escape test passed}\or%
  \PackageError{@@PACKAGE}{\string\includeRnw\space requires --shell-escape. Current is restricted shell escape}{}\stop\fi%
}

\def\check@output@dir{%
  \immediate\write18{touch \rnw@check@ouput@file}
  \IfFileExists{\rnw@check@ouput@file}{}{%
    % Create output dir if not exists
    \immediate\write18{mkdir \rnw@dir@output}
  }
  \immediate\write18{rm \rnw@check@ouput@file}
}

\def\rnw@include@knithead{%
  % Only make the file if the knitr header does not exist
  \IfFileExists{\rnw@filebase@knithead.tex}{}{%
    \@dlog[0]{Creating knithead}
    % MakeRnwFile for generating knithead
    \immediate\write18{echo "<<create-preamble,echo=FALSE,results='asis'>>=\string\ncat(knitr:::make_header_latex())\string\n@" > \rnw@filebase@knithead.Rnw}
    %Build texfile via knitr for preamble
    % \immediate\write18{R -e 'library("knitr");knit("\rnw@filebase@knithead.Rnw","\rnw@filebase@knithead.tex")' >> \rnw@file@knitlog}
  }
  \IfFileExists{\rnw@filebase@knithead.tex}{\@dlog[0]{Including knithead}\incl{\rnw@dir@output/knithead.tex}}{\PackageError{@@PACKAGE}{Could not find knitr preamble: \rnw@dir@output/knithead.tex}{}}
  % \immediate\write18{echo "\string\\\string\rnw@dir@output" >\rnw@dir@output/tst}
}
%% gopt: short for given opt
%:§copts
%:Options to use with \refCom{includeRnw}
\xdef\rnw@gopt@halt{h}
%:\optDef{h}{Use the \refKey{h} to prevent \includeRnw from
%:actually knitting the file, but only include the knitted \!texttt{.tex} file}
\xdef\rnw@gopt@force{f}
%:\optDef{f}{\refKey{f} forces \includeRnw to actually knitting the file no matter,
%:what (as long as it exists).}
%:-
\def\ifrnw@should@knitr{
  \@dlog[0]{Deciding whether to do knitting}
  \let\ifrnw@local@build\ifrnw@opt@build
  \newif\ifrnw@doknit
  \rnw@doknittrue
  \@dlog[2]{Decision is given option: \givenopt}
  % Always build on force
  \ifnum\pdfstrcmp{\givenopt}{\rnw@gopt@force}=\z@\relax\rnw@doknittrue\@dlog[2]{Focing build}\else%
    % If given local option halt, it overrides the package option
    % and acts as oposite of force
    \ifnum\pdfstrcmp{\givenopt}{\rnw@gopt@halt}=\z@
        \rnw@doknitfalse\relax\@dlog[2]{Halt option given. Not building.}%
        % \else\relax%
        % If neither force or halt is given then if new, do
        % conditionals:
        % \ifrnw@opt@new%
        %   % \IfFileExists{\knitOutfile}{\rnw@doknitfalse}{\rnw@doknittrue}
        % \else%
        %   % If new not given, then rely on \rnw@opt@build
        %   \@dlog{No new given}
        %   \ea\ea\let\ea\ifrnw@doknit\csname ifrnw@opt@build\endcsname\relax%
        % \fi%
    \fi
  \fi
\ifrnw@doknit
\@dlog[1]{Decided to build}
}

\def\rnw@purge@outdir{
  \immediate\write18{rm -rf \rnw@dir@output}
  \immediate\write18{mkdir \rnw@dir@output}
}

\def\rnw@preKnit{
\newtoks\mytoks
\mytoks{\\}
\def\bs{\@backslashchar}
\def\doublebs{\bs\bs\bs\bs\bs}
\def\fourbs{\doublebs\doublebs\doublebs}
}
% \rnw@execute@knitr{path/to/file.Rnw}. .Rnw can be omitted
\providecommand\rnw@execute@knitr[2][]{%
  \def\givenopt{#1}
  % Set filename structure
  \rnw@filename@parse{#2}
  \xdef\knitOutfile{\rnw@dir@output/\filename@base\rnw@suffix.tex}
  % decide if should knitr
  \ifrnw@should@knitr
    \IfFileExists{\rnw@infile@fullpath}{
      \@dlog[2]{Building \rnw@infile@fullpath\space to \rnw@dir@output/\filename@base\rnw@suffix.tex}
      \rnw@preKnit
      \immediate\write18{\rnw@knitCommand}
    }{\@latex@error{@@PACKAGE: Could not find file that I was asked to knit: \rnw@infile@fullpath!}{}\stop}
  \else \@dlog[2]{- Skipping knit of \rnw@infile@fullpath}
  \fi
  \IfFileExists{\knitOutfile}{%
    \@dlog[2]{Found \knitOutfile. Including it.}
    \incl{\knitOutfile}
  }{%
    % \@latex@error{includeRnw: Couldn't find knitted file: \knitOutfile}{}
    \PackageError{includeRnw}{Couldn't find knitted file: \knitOutfile}{}
  }
}
%% Usage \rnw@settable{\usercommand}{\internal@commmand}{\actionToDo}
\newcommand\rnw@settable[3][]{%
\edef\reserved@R{\expandafter\@gobble\string #2}%
\@ifundefined\reserved@R%
  {% Undefined, we're good
    \gdef#2##1{\gdef#3{##1}#1}
  }%
  {% Already defined
    % Did the user define the \nocommand?
    \@ifundefined{no\reserved@R}{
      % If not, we should trow an error
      \@latex@error{includeRnw: Tried to define \@backslashchar \reserved@R \space, but it's already defined. \MessageBreak If you don't need the command \@backslashchar \reserved@R \space, then just define \string\no\reserved@R\space before you include the includeRnw package: \string\def\string\no\reserved@R{}}{}\stop
    }{
      % It's defined, so the user don't need our \reserved@R.
      % Let's just remind the user that it is defined
      \@latex@warning{\@backslashchar\reserved@R is not defined by includeRnw since it was already defined someewhere else. Continuing since \string\no\reserved@R is defined.}
    }
  }%
}
%
%
% \let\ea\expandafter
% \gdef\tst##1{##1}
% \gdef\rnwInputDirectory##1{\gdef\rnw@dir@input{##1}}
% \ea\gdef\csname #1\endcsname##1{\ea\gdef\csname #2\endcsname{##1}}
%---- User commands ----%
%:§user
%: The default user macro settables are
%:!\rnwInputDirectory{.}
%:!\rnwKnittedSuffix{knitted}
%:!\rnwKnitlogFile{\rnw@dir@output/knitlog.log}
%:!\rnwKnitheadName{\rnw@dir@output/knithead}
%:-
%:Where \rnw@dir@output is the value of the set \rnwOutputDirectory
%:=\rnwInputDirectory{directory}
%: This command sets what directory \includeRnw will be using. If you e.g.
%: have all the Rnw-files inside a directory called \texttt{myRnws} then
%: \rnwInputDirectory\brackets{myRnws} would fix this.
\rnw@settable\rnwInputDirectory\rnw@dir@input
%:=\rnwKnittedSuffix{text}
%: Set the default suffix of the filename of the knitted output
\rnw@settable\rnwKnittedSuffix\rnw@suffix
%:=\rnwKnitlogFile{filename}
%: Sets the filename of the log output by knitr.
\rnw@settable\rnwKnitlogFile\rnw@file@knitlog
%:=\rnwKnitheadName{filename}
%: Sets the filename of the preamble generated by knitr.
\rnw@settable\rnwKnitheadName\rnw@filebase@knithead
%:-
%:§ref
%:=\includeRnw[irw-options]{path/to/file.Rnw}
%:Compiles .Rnw-files using R. Assumes that \texttt{R} can be called
%:from the command line. It is optional to add the extension \texttt{.Rnw}
%:-
\let\includeRnw\rnw@execute@knitr
\let\purgeOutDir\rnw@purge@outdir
\check@shell@escape
\check@output@dir
\rnw@clear@knitlog
\ifrnw@opt@knithead%
% Knithead runs \makeatother
  \rnw@include@knithead\makeatletter
\fi
